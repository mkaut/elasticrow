\documentclass[a4paper]{scrartcl}

\usepackage{elasticrow}
\usepackage{caption}
\usepackage{hyperref}
\hypersetup{colorlinks,allcolors=blue}
%\captionsetup[subfigure]{labelformat=empty}

\title{The \texttt{elasticrow} package}
\author{Michal Kaut}


\begin{document}

\maketitle

\begin{abstract}
	This package introduces a new environment \verb|elasticrow| and command \verb|\elasticfigure| for aligning several images in one row/line so that they have the same height and, at the same time, fill a specified width, with an optionally specified white space between.
	
	It is based on a \href{https://tex.stackexchange.com/q/227935}{StackExchange solution by Herbert Kruitbosch}.
\end{abstract}

\section{Usage}

The general usage is illustrated in the following example:

\begin{figure}[!h]
	\centering
	\begin{elasticrow}[width=0.75\linewidth, sep=1em]
		\elasticfigure[
			caption=Figure with an A,
			label=fig:subfig1
		]{example-image-a}
		\elasticfigure[
			caption=A 16x9 figure,
			label=fig:subfig2
		]{example-image-16x9}
	\end{elasticrow}
	\caption{Caption of the surrounding \texttt{figure} environment}
	\label{fig:fig1}
\end{figure}

\noindent
typeset as
\begin{verbatim}
	\begin{figure}[!h]
	    \centering
	    \begin{elasticrow}[width=0.75\linewidth, sep=1em]
	        \elasticfigure[
	            caption=Figure with an A,
	            label=fig:subfig1
	    	]{example-image-a}
	    	\elasticfigure[
		        caption=A 16x9 figure,
		        label=fig:subfig2
	    	]{example-image-16x9}
	    \end{elasticrow}
	    \caption{Caption of the surrounding \texttt{figure} environment}
	\end{figure}
\end{verbatim}

, where each subfigure can be individually referenced with \verb*|Fig.~\ref{fig:subfig1}|, \verb*|Fig.~\ref{fig:subfig2}|, and \verb*|Fig.~\ref{fig:sub1}|, rendered as Fig.~\ref{fig:subfig1}, Fig.~\ref{fig:subfig2}, and Fig.~\ref{fig:fig1}.

\subsection{Control parameters}

The package exposes two lengths:
\begin{description}
	\item[\texttt{\textbackslash{}elasticrowwidth}] is the width of the rows; it defaults to \verb*|\textwidth|
	\item[\texttt{\textbackslash{}elasticrowsep}] is the white space between figures; it defaults to \verb*|0pt|
\end{description}

\subsection{Commands and environments}

Each figure is placed using the \verb*|\elasticfigure[<caption>]{<figure_file>}| command, where \verb*|<figure_file>| is a file name of the included figure (which will be passed to \verb*|\includegraphics{}|) and \verb*|<caption>| is an optional caption.

All \verb*|\elasticfigure|s that should be on one line have to be put inside an \verb*|elasticrow| environment.
Its opening syntax is \verb*|\begin{elasticrow}[<options>]|, where the optional \verb*|<options>| is a key-value string with allowed keys \verb*|width| and \verb*|sep|, overriding the default lengths \verb*|\elasticrowwidth| and \verb*|\elasticrowsep|, respectively.

\subsection{Notes}

\verb*|\elasticfigure| puts figures into a \verb*|\subcaptionbox{}| only if they have captions, otherwise it uses a pure \verb*|\includegraphics{}| command.
This means that if all figures are without captions, they can be placed outside of \verb*|figure| environment:

\begin{center}
	\begin{elasticrow}
		\elasticfigure{example-image-a}
		\elasticfigure{example-image-b}
		\elasticfigure{example-image-c}
		\elasticfigure{example-image-golden}
	\end{elasticrow}
\end{center}
typeset using the default values of \verb*|\elasticrowwidth| and \verb*|\elasticrowsep| as
\begin{verbatim}
\begin{center}
    \begin{elasticrow}
        \elasticfigure{example-image-a}
        \elasticfigure{example-image-b}
        \elasticfigure{example-image-c}
        \elasticfigure{example-image-golden}
    \end{elasticrow}
\end{center}
\end{verbatim}

\end{document}
